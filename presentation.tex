\documentclass{beamer}
\usepackage[utf8]{inputenc}
\usepackage[greek,english]{babel}
\usepackage{cite}

\title[Online Learning]{Online Learning for Min Sum Set Cover and Pandora's Box}
\author{Evangelia Gergatsouli, Christos Tzamos \\ Presented by: Ioannis Kasionis, Petros Triantafyllos}
\date{January 2025}

\begin{document}

% Title Slide
\begin{frame}
  \titlepage
\end{frame}

% Slide 1: Introduction
\begin{frame}{Introduction}
\textbf{Context:} Stochastic optimization problems\\
\textbf{Focus:} Online version of Min Sum Set Cover (MSSC) and Pandora's Box\\
\textbf{Key Question:} How to minimize regret in online settings?
\end{frame}

% Slide 2: Problem Definitions
\begin{frame}{Problem Definitions}
\textbf{Pandora’s Box:}
\begin{itemize}
    \item Selection with unknown costs.
    \item Goal: Minimize selection and exploration costs.
\end{itemize}
\textbf{MSSC:}
\begin{itemize}
    \item Minimize the weighted sum of covering times for scenarios.
\end{itemize}
\textbf{Online Challenges:}
\begin{itemize}
    \item Adversarially chosen scenarios.
    \item Regret minimization.
\end{itemize}
\end{frame}

\begin{frame}{Methodology}

    \textbf{Methodology Outline:}
    \begin{enumerate}
        \item Scenario-aware convex relaxation.
        \item Online Convex Optimization (e.g.\ FTRL, explore--exploit).
        \item $\alpha$-approx.\ rounding, scenario-independent.
    \end{enumerate}
    
    \textbf{Goal:} Achieve \emph{no-regret} (or $\alpha$-approx.\ no-regret) against adversarial inputs, matching near-optimal offline constants.
    \end{frame}
    
    % Slide 2: Algorithm as (sigma, tau)
    \begin{frame}{Describing Policies via $(\sigma, \tau)$}
    \textbf{Any algorithm can be described} by:
    \[
      (\sigma, \tau),
    \]
    where
    \begin{itemize}
      \item $\sigma$ is a permutation of the boxes, representing the \emph{order} in which they are opened.
      \item $\tau$ is a \emph{stopping rule}, deciding when to stop opening further boxes and select the best seen.
    \end{itemize}
    
    \end{frame}
    
    % Slide 3: Main Contribution from [CGT+20]
    \begin{frame}{Key Result from \cite{CGT+20}}
    \textbf{Simplicity via Ordering:}\\
    One major insight in \cite{CGT+20} is that \emph{it suffices to design a strategy} that:
    \begin{itemize}
        \item Chooses \emph{an ordering} of boxes to open,
        \item Performs well under the assumption we know \emph{when to stop}.
    \end{itemize}
    
    \vspace{0.5em}
    \textbf{Why is this powerful?}
    \begin{itemize}
        \item We do not need to handle all possible dynamic reorderings.
        \item Scenario-awareness plus a single (or fixed) ordering handles partial adaptivity with small overhead.
    \end{itemize}
    
    \textbf{Thus:} The combinatorial explosion of fully adaptive policies is avoided. We focus on \emph{permutation} + \emph{stopping} rather than re-choosing order after each cost revelation.
    \end{frame}
    
    
    % Slide 5: Why SPA and How Rounding Escapes Scenario Dependency
    \begin{frame}{Scenario-Aware Partially Adaptive (SPA)}
    \textbf{SPA:} We fix an ordering for boxes and a possible stopping point, but \emph{adapt} which boxes to open in each round \emph{based on} partial observations.
    
    \textbf{Rationale:}
    \begin{itemize}
        \item Fully adaptive strategies are intractably large.
        \item \emph{Scenario-aware} partial adaptivity is near-optimal (loses only a constant factor).
    \end{itemize}
    
    \textbf{Rounding to Escape Scenario Dependence:}
    \begin{itemize}
        \item Solve a fractional relaxation that depends on scenario $s$.
        \item \emph{One rounding procedure} transforms that fractional $x$ into an integral plan \emph{without} re-optimizing for $s$.
        \item Ensures a constant-factor overhead $\alpha$ and preserves no-regret in an adversarial setting.
    \end{itemize}
    \end{frame}
    
    \begin{frame}
    \vspace{0.5em}
    \textbf{Lemma 2.1 (Simplification of Theorem 3.4 from \cite{CGT+20}):}\\
    For any polynomial-time $\alpha$-approx.\ algorithm for \emph{scenario-aware partially adaptive} strategies, there is a polynomial-time algorithm that is 
    \[
     \frac{e}{e-1}\,\alpha\text{-approx.}
    \]
    as a partially adaptive strategy. 
    
    \textbf{Interpretation:} Reduces fully adaptive policies to carefully chosen permutations $+$ a robust stopping rule, paying only a constant factor $\frac{e}{e-1}\approx 1.58$ overhead.
    \end{frame}
    
    % Slide 6: Methodologies (OCO + Explore-Exploit)
    \begin{frame}{Online Methodologies: OCO + Explore--Exploit}
    \textbf{Full-Information vs.\ PA:}
    \begin{itemize}
      \item Observe entire cost function each round, apply FTRL (Follow-the-Regularized-Leader).
      \item Round solutions scenario-independently.
      \item Achieves $\alpha$-approx.\ no-regret \emph{e.g.\ 9.22 for 1 box, $O(1)$ for $k$ boxes.}
    \end{itemize}
    
    \textbf{Bandit vs.\ PA:}
    \begin{itemize}
      \item Partial feedback: only see the cost at your chosen fractional point.
      \item Occasionally \emph{open all boxes}, paying cost $n$, to obtain full function info (exploration).
      \item Combine partial info + full info to keep regret sublinear, maintaining the same $\alpha$-approx.\ factor.
    \end{itemize}
    \end{frame}
    
    % Slide 7: PA Has the Same No-Regret Guarantees
    \begin{frame}{Partially Adaptive No-Regret Guarantees}
    \textbf{Why the same approximation factors remain?}
    \begin{itemize}
        \item Rounding is independent of scenario once the fractional solution is found.
        \item For full-info, FTRL yields sublinear regret in the fractional domain.
        \item For bandit, enough full exploration to approximate the loss + partial feedback in normal rounds.
    \end{itemize}
    
    \textbf{Examples of Achieved Bounds:}
    \[
    \begin{cases}
    9.22\text{-approx no-regret for single box selection,}\\
    O(1)\text{-approx no-regret for $k$ boxes,}\\
    O(\log k)\text{-approx no-regret for matroid constraints.}
    \end{cases}
    \]
    \textbf{MSSC Special Case:} The 4-approx.\ from offline is matched in the online adversarial sense.
    \end{frame}
    
    % Slide 8: Results for Non-Adaptive (NA) and Conclusion
    \begin{frame}{Results for NA \& Conclusion}
    \textbf{NA Setting:}
    \begin{itemize}
        \item Non-Lipschitz costs block direct use of FTRL.
        \item Explore--exploit subroutine: solve a global LP + rounding.
        \item Yields no-regret with constants e.g.\ 3.16 (1 box), 12.64 ($k$ boxes), $O(\log k)$ (matroids).
    \end{itemize}
    
    \textbf{Conclusion:}
    \begin{itemize}
        \item Using scenario-aware partial adaptivity + robust rounding avoids scenario dependence.
        \item OCO or explore--exploit methods yield constant-approx.\ no-regret for Pandora’s Box / MSSC.
        \item The $\frac{e}{e-1}\alpha$ overhead from \cite{CGT+20} ensures polynomial-time solutions with near-optimal adaptivity.
    \end{itemize}

% Slide 5: Bandit Setting
\begin{frame}{Bandit Setting}
\textbf{Key Features:}
\begin{itemize}
    \item Limited feedback: Only revealed values for opened boxes.
    \item Approximation guarantees similar to the full information setting.
    \item Practical for real-world scenarios.
\end{itemize}
\end{frame}

% Slide 6: Comparison with Previous Work
\begin{frame}{Comparison with Previous Work}
\textbf{Improvements over [CGT+20]:}
\begin{itemize}
    \item Simpler algorithms.
    \item Broader applicability.
\end{itemize}
\textbf{Advances over [FLPS20]:}
\begin{itemize}
    \item Extends to bandit settings.
    \item Handles more complex constraints.
\end{itemize}
\end{frame}

% Slide 7: Applications and Open Questions
\begin{frame}{Applications and Open Questions}
\textbf{Applications:}
\begin{itemize}
    \item Resource allocation.
    \item Decision-making under uncertainty.
\end{itemize}
\textbf{Open Problems:}
\begin{itemize}
    \item Tight bounds for MSSC.
    \item Extensions to dynamic settings.
\end{itemize}
\end{frame}

% Slide 8: Conclusion
\begin{frame}{Conclusion}
\textbf{Summary of Contributions:}
\begin{itemize}
    \item Framework for online learning with MSSC and Pandora’s Box \cite{gergatsouli2022online}.
    \item Approximation guarantees for various settings.
    \item Computationally Efficient Algorithms.
\end{itemize}
\textbf{Acknowledgments:}
\begin{itemize}
    \item Authors: Evangelia Gergatsouli, Christos Tzamos.
\end{itemize}
\end{frame}

% Bibliography Slide
\bibliographystyle{unsrt}
\bibliography{report}
\end{document}
